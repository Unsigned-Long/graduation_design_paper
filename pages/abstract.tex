% 中文摘要

多源融合能够弥补单一传感器的不足,在复杂环境中提供更可靠的服务,因此被广泛接受为一种处理导航和感知任务的可行解决方案。对于一个多传感器系统而言,想要确保融合算法正常工作,准确的时空参数是必要的,因为不准确的时空参数会导致不同传感器之间的不一致,从而损害融合的性能。为此,我们提出了一种基于连续时间轨迹估计的无靶标LiDAR/Camera/IMU时空标定方法,其使用由环境特征构建的约束来估计空间和时间参数,不依赖目标或者先验知识。为了将时间参数的估计与空间参数的估计紧密地结合在一起,我们使用B样条曲线来建模连续时间轨迹。与离散时间方法相比,这种连续时间表示方法可以达到更高的精度,尤其是在高度动态场景下。为了研究不同退化运动对标定精度的影响,我们基于李导数,对系统的可观性进行了理论分析和实验验证,并证明了整个系统在随机运动下是完全可观的。为了验证所提出方法的可行性并评估其性能,我们进行了仿真实验和实测实验。结果表明,我们的方法能够在保证系统一致性的前提下,实现高精度的时空参数标定。
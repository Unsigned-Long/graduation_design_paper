% 中文摘要

在位置服务和场景重建领域,多传感器融合作为一种解决问题的有效手段,能够充分利用不同传感器之间的互补特性,在复杂场景下提供更优、更可靠的服务。对于一个多传感器系统而言,时空参数的正确标定是多传感器融合算法能够正确运行的前提条件,且相关参数的标定精度直接影响着算法的性能。为此,我们提出了一个基于连续时间的LiDAR/Camera/IMU的无靶标时空标定方法。该方法利用环境特征构造约束来估计参数,不依赖靶标和先验知识,且同时考虑了多传感器平台空间参数和时间参数的标定。特别地,该方法基于连续时间理论,利用B样条曲线建模位姿轨迹,将时参的估计严密地纳入到因子图中,在高动态场景下相较于离散时间估计方法有着更高的标定精度。随后,我们基于李导数,对不同形式运动下的系统可观性进行了理论分析和实验验证,证明了在随机运动下,整个系统是完全可观的。为了验证所提出方法的可行性并评价其性能,我们进行了仿真实验和室内外实测实验,结果证明我们的方法能够在保证一致性的前提条件下达到较优的时空参数标定精度。
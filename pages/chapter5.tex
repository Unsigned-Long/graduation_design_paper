\chapter{\normf{总结和展望}}
\section{\normf{论文工作总结}}
高精度的场景重建和连续、可靠 、无缝的位置服务已成为产业应用的迫切需求。在场景重建时,往往需要利用多个传感器的互补特性进行多传感器融合,以恢复出一致、丰富、有效的场景;在复杂场景提供位置服务时,单一传感器性能退化,无法满足需求,同样需要通过多传感器融合来实现。而鲁棒的多传感器融合算法的实现依赖于多个传感器之间相关参数的精确标定。本文在多源融合的大背景下,从多传感器标定角度切入,对基于连续时间的LiDAR/Camera/IMU的时空标定算法进行了深入的研究。本文首先对连续时间估计涉及的必要理论知识进行了阐述,接着从初始化、数据关联和批处理优化三个部分出发,对算法进行了详细的阐述。为验证所提出的基于连续时间的LiDAR/Camera/IMU的时空标定方法的可行性,我们设计了仿真实验,并分析了相关参量的绝对误差曲线。而后,我们基于自主搭建的实验设备进行了实测实验,在室内外两种不同的典型场景下,对算法进行综合的测试分析,验证了算法的鲁棒性。最后,我们对系统的可观性进行了理论分析和实验验证。现将本文的主要工作总结如下:
\begin{enumerate}
  \item 本文基于连续时间理论,提出了一种无靶标的多传感器标定框架,并使用CPP语言对其进行了实现\footnote{\normf{Github链接:\url{https://github.com/Unsigned-Long/LIC-Calib.git},目前暂未开源。}}。在系统可观的条件下,多传感器内参、外参和时延等时空参数都能在该标定框架标定。相比于传统有靶标的标定方法,该方法具有易操作、易推广的优点;相比于离散时间估计,该方法具能够保证时参估计的可靠性。

  \item 基于提出的时空标定算法,我们对算法的可行性和性能进行了综合的测试和评估。我们设计了相应的仿真实验和室内外实测实验,并对结果进行统计分析,结果表明:在系统可观的条件下,只需要少量次数的迭代,时参估计精度能够达到$0.1\;ms$,姿态量估计精度能够达到$0.1°$,位移量估计精度能够达到$5\;mm$,具有较好的精度和计算效率。

  \item 为了评估算法的一致性,我们基于解算完成后的IMU位姿B样条曲线、LiDAR点云地图和相机图像序列,重新构建带有灰度信息的三维场景。结果表明:标定过程中整个系统是一致的,不同约束之间通过IMU位姿B样条进行联系,能够保证解算结果的最优和一致。

  \item 为了考察不同运动形式下系统的可观性,我们基于LM方法的增量方程进行了系统可观性的理论分析。同时,我们依托仿真平台进行测试实验,并对结果进行了整理分析。结果表明:在运动充分激励的条件下,整个系统的参数是可观性的,且能够保证参数的估计精度;而在退化运动形式下,系统的部分参数不可观;在无运动条件下,系统完全不可观。
\end{enumerate}
\section{\normf{研究展望}}
在本文研究算法的基础上,今后将在以下方面开展进一步工作:
\begin{enumerate}
  \item 当前系统基于的是图优化算法进行参数估计,其能够通过多次迭代消除非线性系统线性化引入的误差,但是同时也损失了相应的计算效率。事实上,在解算之前,可以基于一定的数据筛选策略,在保证不损失系统精确度和可靠性的前提下,过滤掉一些信息含量少的数据,以减少解算的数据量,从而提高系统的运算效率\cite{lv2022observability}。

  \item 当前标定系统对于相机内参的处理采取的是一种由粗到细的解算策略:在手动给定初值(尤其是相机焦距)的条件下,通过迭代优化,对相机内参进行不断精化。该种处理方式虽然简单有效,但需要额外对相机内参进行粗略的标定,破坏了系统的完整性。事实上,可以将带有尺度信息的LiDAR传感器和相机进行联合标定,通过基于目标或几何特征的方法进行约束构建,实现相机内参的初始化。当然,该工作的引入可能会让标定系统变得庞大臃肿。

  \item 本文的提出的标定系统目前支持的是不同类型单传感器之间的标定(一个相机、一个LiDAR,一个IMU)。对于一个多类型多传感器(如多相机、多IMU、多LiDAR)的系统而言,在使用传统的方法进行标定时,往往需要进行多批次的处理。这不仅操作繁琐,而且会破坏系统标定结果的一致性。针对这一点,可以对标定系统进行相应的扩展,以满足上面的需求。另外,当前系统支持的相机是针孔相机,LiDAR为机械式激光雷达。而目前鱼眼相机和其他类型的雷达(如Livox)也在多源融合领域有所应用,后续可以针对这一点进行相应的补充和完善。

\end{enumerate}

\newpage
\appendix

% Chapter 1

\chapter{\normf{绪论}}
\normf
近年来,随着自动驾驶、服务机器人等智能技术的发展,位置服务和场景感知变得越来越重要。比如,智能车辆在行进过程中,需要进行连续、可靠 、无缝的高精度定位;机器人在运作过程中,需要进行实时定位和建图,同时感知周围的环境。对于简单的应用场景,单一传感器即可胜任上述工作。但在复杂场景下,基于单一传感器的技术存在局限性,无法满足上述需求,往往需要通过多传感器融合技术来实现。以无人车为例,在城市峡谷、隧道等复杂场景下,基于单一传感器的技术很难为无人车提供连续、可靠的定位,而通过多传感器的融合即可以达到这一需求;在人流、车流密集的城市内部道路中,无人车需要对周边复杂的场景进行感知以作出相应的决策,对此单一传感器无法胜任,需要通过多个传感器配合来实现。相较于单一传感器系统,多传感器组合系统能够发挥不同传感器之间的互补特性,在复杂场景中提供更优的服务。但是,多传感器组合系统需要额外进行多传感器之间相关参数的标定,且参数标定的精确程度直接影响着算法实现高精度定位或精准场景感知的能力。

多传感器标定算法因传感器类型而异。目前在自动驾驶和智能机器人领域,使用得较多的传感器是LiDAR、Camera和IMU,它们都能独立地提供载体相对于参考坐标系的位姿信息,且存在各自的优势和不足,因此常常通过融合算法组合使用。相应的,也就需要对组合系统涉及的多个传感器进行标定。基于这一点,本章先对涉及上述三类传感器的多传感器标定算法的研究现状和发展趋势进行分析,而后引出本文的具体研究内容。

\section{\normf{研究背景与意义}}
就位置服务而言,在目前存在的各种定位技术中,全球卫星导航系统(Global Navigation Satellite System,GNSS)无疑是应用得最为广泛的全局定位技术,其能够在全球范围和近地表面提供全天候、全天时的定位服务。但是,GNSS技术作为一种基于无线电信号的导航定位技术,在诸如城市峡谷、隧道、室内等信号易受遮蔽和干扰的复杂场景中,其定位精度会有所下降,甚至完全失效\cite{li2023multi}。

相较于易受外界环境干扰的GNSS技术,基于惯性测量单元(Inertial Measurement Unit,IMU)实现的惯性导航系统(Inertial Navigation System,INS)不受环境的干扰,能够通过连续积分实现相对定位,具有高频、自主、无源的优点,但是受测量噪声和零偏的影响,其不可避免地存在误差发散问题。因此在车载导航中,常常将两者结合使用,利用GNSS提供的全局定位信息抑制INS的误差发散,同时在复杂环境中通过INS辅助GNSS定位,以提供更加连续、可靠的位置服务\cite{titterton2004strapdown}。IMU除了在车载导航中被广泛使用外,其作为一种高频、低功耗的内部传感器,在近年来不断发展的诸如增强现实(Augmented Reality,AR)、智能机器人等新兴技术领域,也有着较大的应用潜能。

相较于IMU这种内部传感器,Camera、LiDAR等外部传感器可以在提供载体位姿的同时感知周围的环境,因此被广泛应用于自动驾驶和智能移动机器人领域。Camera可以对周围场景连续地获取二维影像。基于所获取影像的几何特征和光度信息,载体可以在实现定位定姿的同时,完成场景的识别、扫描和重建。但是,相机在成像过程中将外部的三维场景映射到二维影像平面上,导致了场景深度信息的缺失。因此对于单目相机而言,即使通过相应的算法恢复了场景结构,其尺度也是未知的。为了恢复尺度,可以通过组合多个相机来实现,也可以通过结合LiDAR这种带有尺度的传感器来达到。相较于Camera,LiDAR能够直接感知周围环境的距离和方位信息,但无法获取场景的纹理信息。因此可以将LiDAR和Camera结合使用,以充分发挥两类传感器之间的互补特性,实现更为鲁棒的相对位姿估计和场景重建。同样的,IMU也可以辅助Camera和LiDAR,进行特征跟踪和畸变消除,同时提供相对位姿约束。

总的来说,多传感器融合作为一种解决复杂问题的有效手段,在位置服务、场景感知或是其他智能服务领域都被广泛地使用。但是相应的,也需要进行额外的多传感器标定工作。对于大多数的多传感器融合算法而言,多传感器系统被正确地标定是实现精确的运动估计或有效的场景感知的前提。相比于只需关注传感器内参的单传感器算法,多传感器算法为了融合来自不同传感器的量测信息,需要考虑各传感器之间的外参。另外,粗糙的时钟同步、不精确的硬件触发以及传输延迟,也会导致系统的不稳定或状态估计不准确,因此与时间相关的参数也需要校准\cite{yang2019degenerate}。特别的,对于卷帘快门相机(Rolling Shutter,RS)而言,由于其以单行扫描的方式进行图像采集,不同像素行之间存在时差。在动态场景下,这会导致图像受果冻效应的影响,因此与此相关的时间参数也需要被标定。

基于上述需求,本文提出了一个基于连续时间(Continuous Time,CT)的无靶标多传感器时空标定框架,并基于LiDAR/Camera/IMU组合系统对其进行了实现。相较于基于离散时间(Discrete Time,DT)的标定方法,其将时参严密纳入到最优化问题中进行求解。该标定框架具有易操作性、可扩展性和鲁棒性,对于多传感器融合算法的研制和相关智能服务设施的正真落地有着重要的实际意义和应用价值。

\section{\normf{国内外研究现状和发展趋势}}

\subsection{\normf{Camera/IMU方面}}
Camera/IMU组合系统使用的是低成本、低功耗的Camera和IMU。Camera可以连续地获取场景的灰度影像,能够在提供较高精度相对位姿的同时恢复出场景的三维结构;IMU可以连续、高频地输出载体相对于惯性坐标系的线加速度和角速度,能够为Camera的相对位姿估计提供预积分量测(Pre-Integration Measurement,PIM)约束\cite{qin2018vins},或者在视觉失效的场景下能够连续独立地输出载体的相对位姿信息。因此诸如视觉惯性里程计(Visual Inertial Odometry,VIO)、视觉辅助的惯性导航系统(Vision-aided Inertial Navigation Systems,VINS)等Camera/IMU组合系统在增强现实、智能机器人、文物保护和考古等领域有着较为广泛的应用。而且出于成本的考虑,这些部署在诸如智能手机、扫地机器人等智能设备上的Camera/IMU组合系统一般配备的都是价格低廉的Camera和MEMS级别的IMU,在出厂时传感器相关参数的标定都较为粗略,且到消费者手中后几乎不可能进行二次标定(尤其是基于有靶标的标定方法),显然这会严重损害相关算法的精度。为此,Huai等\cite{huai2022observability}基于滤波方法实现了一个能够进行在线自标定的VIO系统,并对系统的可观性进行了分析。该方法在标定过程中同时考虑了传感器的内参、外参和时参,是一个基于自运动进行无靶标标定的方法。Mirzaei等\cite{mirzaei2008kalman}基于迭代扩展卡尔曼滤波器(Iterated Extended Kalman Filter,IEKF),实现了一个事后的Camera/IMU标定系统,但是其需要通过标定板初始化相机的位姿序列,是一种有靶标的标定方法。另外,该方法需要通过CAD或人为量测来给定待标参数较为精确的初值,而这在实际操作中难以保证,因此其应用受较大程度上的限制。与此类似,Yang等\cite{yang2019degenerate}基于多状态约束下的卡尔曼滤波器(Multi-State Constraint Kalman Filter,MSCKF)实现了一个Camera/IMU标定系统,并在理论层面考察了不同形式的退化运动对系统可观性的影响。

上述基于滤波的标定方法都依托离散时间估计理论对待标参数进行估计,在处理传感器异步测量值时需要将所有状态递推到测量时刻,且当涉及时参估计时,该问题会变得更为严峻。相反的,连续时间估计理论通过构建时间连续函数,将时参纳入到严密的估计系统中,在处理传感器异步测量值时,只需要进行简单的时间采样(Sampling)或查询(Querying),即可获得对应的系统状态。Furgale等\cite{furgale2013unified}基于连续时间估计理论,使用B样条曲线建模IMU的位姿和零偏,实现了一个基于图优化估计的Camera/IMU标定系统。该方法将Camera和IMU之间的时延参数纳入到因子图中进行估计,在给定较为精确的传感器量测模型下,能够获得较高精度的估值。但是,该方法需要通过标定板初始化相机的位姿序列,也同样需要较为精确的外参初值,以基于初始化的相机位姿序列恢复出初始的IMU位姿序列。之后,他们将激光雷达也纳入到该系统中进行标定\cite{rehder2014spatio}。

由上文可知,当前的Camera/IMU标定方法大部分基于离散时间估计,通过滤波方法进行实现。在涉及时参估计时,该类方法实现的系统的复杂度将会大大增加,同时性能会降低。而基于连续时间估计,通过图优化方法实现的标定方法则能较好的处理时参估计问题。且由于使用图优化进行参数估计,此类标定方法往往需要一个初始化的步骤来提供待估参数的初值,但目前大部分的标定系统的初始化不是自动完成的,而是通过手动赋值或者靶标辅助的方式进行,这在大多数情况下是不能满足的。

\subsection{\normf{LiDAR/IMU方面}}
LiDAR/IMU组合系统使用的传感器为激光雷达(Light Detection and Ranging,LiDAR)和IMU。与Camera单帧只能获取场景在像平面上的二维投影不同,LiDAR单帧可以直接获取场景的三维信息。为了重构完整的场景,Camera需要利用相机影像序列构建多视图几何问题,并耗费大量计算资源来求解方程组,且对于单目相机而言,场景的尺度是未知的。而LiDAR只需要进行简单的帧到帧或者帧到地图匹配即可得到带有尺度的场景点云地图。在位姿估计方面,Camera需要基于特征点法或者光流法,通过共视图构建优化问题求解,且由于相机量测模型稍复杂,因此优化问题的非线性较严重\cite{高翔2017视觉}。而LiDAR的量测模型简单,通过匹配即可获得高精度的相对位姿。由于LiDAR能够在输出高精度的相对位姿信息的同时提供高精度的稠密点云地图,因此其在自动驾驶领域颇受欢迎。但在成本方面,LiDAR相较于Camera存在劣势,且会受到运动畸变的影响。对于LiDAR的运动畸变,VICP方法\cite{hong2010vicp}假设运动为匀速运动,在匹配的同时估计LiDAR的速度以去除畸变的影响,但是在低采样频率情况下匀速运动假设很难成立。另一种更常用的方法是使用IMU辅助,基于IMU输出的高频角速度和线加速度信息来进行畸变去除\cite{shan2020lio}。另外,基于IMU输出构建的PIM可以为LiDAR匹配提供更好的初值和更强的约束。因此诸如雷达惯性里程计(Lidar Inertial Odometry,LIO)等的LiDAR/IMU组合系统相较于单LiDAR系统,应用更为广泛。相应的,LiDAR/IMU组合系统的标定也成为了一个问题。Liu等\cite{liu2019novel}基于卡尔曼滤波器(Extended Kalman Filter,EKF)方法,通过提取环境中的球特征、圆柱特征和面特征构建约束,实现了一个无靶标的LiDAR/IMU标定系统。但是,该方法需要使用精确的控制平台来驱动IMU,因此使用受到极大的限制,尤其是在机器人领域。而且该方法使用离散时间估计,标定过程中只考虑了LiDAR/IMU组合系统的内外参数,不涉及时参。

如前文所述,Rehder等\cite{rehder2014spatio}在基于连续时间估计的Cmaera/IMU标定系统\cite{furgale2013unified}基础上引入了激光雷达,提出了一个LiDAR/Camera/IMU组合系统的两步标定方法。该方法在进行LiDAR/IMU标定时,通过面特征构建点到面约束进行参数优化,能够同时估计内外参和时参,但是需要较好的外参初值。Li等\cite{li20213d}同样基于连续时间估计,通过高斯过程(Gaussian Process,GP)建模IMU位姿轨迹,实现了一个无靶标的LiDAR/IMU标定系统。由于加入了初始化模块,其不要参数初值的给定。但是该方法需要事先通过地面激光扫描仪(Terrestrial Laser Scanner,TLS)获取一个稠密的先验点云地图,用以构建点到面关联,而这在实际使用时很难保证。Lv等\cite{lv2020targetless}通过雷达里程计初始化点云地图,并基于B样条曲线构建IMU的位姿轨迹,实现了一个无需靶标的LiDAR/IMU组合系统时空标定框架。且该方法精细设计了一个初始化模块,因此不要事先给定参数初值。之后,他们在可观性感知方面对系统做了进一步优化,使系统在标定时启发式地选择富含信息量的数据片段参与运算,以在不影响标定精度的前提下提高系统的运算效率\cite{lv2022observability}。

如上文所述,目前在LiDAR/IMU系统的标定方面,大部分方法是采用无靶标的方式进行,通过提取环境特征构建约束来求解待估参数。但是,在特征提取和约束构建方面仍然存在一定的问题,大部分方法在一般场景下很难提取到稳健的环境特征来构建约束。另外,基于图优化进行参数估计的标定方法的初始化步骤仍然值得讨论和改进。

\subsection{\normf{Camera/LiDAR方面}}
Camera/LiDAR组合系统使用的传感器为Camera和LiDAR,二者均属于外部传感器。LiDAR能够直接获取周围场景的点云量测信息,这是相机不能做到的;相机可以获取场景的纹理、色彩和表面信息,这是LiDAR不能做到的。由于二者存在较好的互补特性,因此该组合被广泛应用于移动机器人、自动驾驶等领域。如在场景重建方面,Camera/LiDAR组合可以重建出带有纹理的三维场景;在目标检测方面,基于相机拍摄的影像,可以通过识别算法定位目标方向,而后结合LiDAR帧可以快速锁定目标位置。事实上,目前已有RGBD相机实现了类似于Camera/LiDAR组合系统的功能,但是其视场角小,有效测程短,很难在大尺度的场景中发挥作用\cite{高翔2017视觉}。对于Camera/LiDAR组合系统的标定而言,目前提出的方法大多通过关联相机帧和LiDAR帧中的特征构建约束,而后将约束和待标参数一同加入到优化问题中进行求解。Park等\cite{park2014calibration}使用一个多边形平板作为靶标,通过在LiDAR帧和相机帧中分别提取平板顶点,而后构建点到点关联约束求解外参。Li等\cite{li2022accurate}基于自制的带孔棋盘平板,在LiDAR帧和相机帧中估计圆孔中心,而后同样构建点到点关联约束进行求解。该标定方法支持多种常见的LiDAR传感器。

以上所提到的方法均是基于人工靶标实现的系统外参标定。对此,Mishra等\cite{mishra2020experimental}对几种比较典型的基于人工靶标的Camera/LiDAR组合系统的标定方法进行了测试。相比于基于人工靶标的标定方法,基于特征的标定方法利用环境信息构建关联约束进行标定参数的求解,其不需要额外的靶标进行辅助,具有更好的适用性。Yuan等\cite{yuan2021pixel}通过提取环境中的深度连续线特征构建关联关系,实现了Camera/LiDAR组合系统外参的无靶标标定。为了能够精确提取到LiDAR帧中的线特征,该方法需要通过Livox类型的固态雷达静态采集一段时间的数据,以获取较为稠密的点云。Gong等\cite{gong20133d}基于环境中较为常见的三面相交特征提取特征面,结合自运动估计方法构建非线性最小二乘问题进行系统外参的无靶标标定。这些方法都是基于离散时间估计的标定方法,在标定过程中都没有考虑时延参数的估计。Park等\cite{park2020spatiotemporal}基于连续时间估计,通过构造投影因子优化求解LiDAR和Camera时间的外参和时延参数,同样是一种基于自运动的无靶标标定方法。但是其使用简单的一阶内插来构建时间连续轨迹,当载体充分激励运动时会引入较大的偏差(基于自运动的标定方法往往需要运动被充分激励来保证待估参数的精度)。

上文所诉的标定方法虽然都能达到比较好的外参标定精度,但是除\cite{park2020spatiotemporal}外都是静态标定,没有考虑时参,而这对于一个多传感器组合系统是不容忽视的。特别是在高动态场景下,未进行严格的时间同步会造成算法精度的损失,这是不能接受的。

\section{\normf{研究内容及论文总体安排}}

\subsection{\normf{研究内容}}
围绕上述多传感器标定方面的研究现状和发展趋势,本文的研究内容如下:
\begin{enumerate}
    \item 基于连续时间的LiDAR/Camera/IMU的时空标定方法的设计与实现
    
    从连续时间估计理论出发,依托IMU的输出构造位姿B样条曲线,基于多传感器原始观测值构建约束,通过因子图优化方法迭代求解待标定的时空参数(内参、外参、时参),逐步达到系统的全局最优。

    \item 基于连续时间的LiDAR/Camera/IMU的时空标定方法的验证与评价
    
    基于虚拟实验平台构建仿真场景,对所提出的标定框架的可行性进行验证。同时基于自主搭建的实验平台,在室内外典型场景下进行实测实验,对所提出的标定框架进行综合的测试和评估。最后对算法的一致性进行评价。
    
    \item 基于连续时间的LiDAR/Camera/IMU的时空标定方法的系统可观性分析和验证
    
    从LM方法的增量方程出发,通过考察方程系数矩阵的零空间,对不同形式退化运动下的系统待标参数的可观性进行了理论分析。同时通过虚拟实验平台构建模拟轨迹,对系统可观性的理论分析进行验证。
    
\end{enumerate}

\subsection{\normf{章节安排}}

第一章为绪论部分,介绍了多传感器标定的背景与意义,以及目前国内外的研究现状和发展趋势。其中,着重分析了当前针对Camera/IMU、LiDAR/IMU和Camera/LiDAR等多传感器系统的标定算法的特点和局限性,并对本文所研究的基于连续时间的LiDAR/Camera/IMU的时空标定方法进行了必要性分析。

第二章为多传感器标定的基础。本章首先介绍了本文的标定方法涉及的坐标系,以及坐标系之间的刚体变换,接着介绍了LiDAR、Camera、IMU的传感器模型。由于本文研究的是基于连续时间的标定算法,因此本章对连续时间理论也进行了相应的介绍。最后对图优化理论进行了介绍。

第三章详细介绍了本文所提出的基于连续时间的LiDAR/Camera/IMU的时空标定方法。本章首先对整个系统的算法框架和状态定义进行了阐述,而后依据算法框架的处理流程,依次介绍了系统的初始化、数据关联和批处理优化。

第四章对实验部分进行了阐述和分析。在本章中,首先基于仿真实验平台GAZEBO模拟的数据,对所提出的标定框架的可行性进行了验证。而后基于自主搭建的实验平台,在室内外两种不同的典型场景下进行了实测实验,对所提出的基于连续时间的LiDAR/Camera/IMU的时空标定方法进行综合的测试和评估。最后,针对不同类型的退化运动,对系统可观性进行了理论分析和实验验证。

第五章为总结与展望。本章首先基于所提出的多传感器标定算法框架,对当前所完成的工作进行了总结归纳,接着对后续可开展的工作进行了展望。

